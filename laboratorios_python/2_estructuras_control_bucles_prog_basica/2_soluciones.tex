\documentclass[a4paper,12pt]{article}
\usepackage[T1]{fontenc}
\usepackage[utf8]{inputenc}
\usepackage[spanish]{babel}
\usepackage{amsmath}

\title{Soluciones a los ejercicios de\\
Condicionales y bucles en Python (con NumPy)}
\author{}
\date{}

\begin{document}
\maketitle

% ==========================================================
\section*{Ejercicio 1 (for): Mostrar los elementos de un vector}

Dado el vector $v=(1,2,3,4,5)$, escribe un programa que muestre por pantalla todos sus elementos usando un bucle \texttt{for}.

\begin{verbatim}
import numpy as np

v = np.array([1, 2, 3, 4, 5])

for i in range(len(v)):
    print(v[i])
\end{verbatim}

% ==========================================================
\section*{Ejercicio 2 (for): Contar elementos positivos}

Dado el vector $v=(-2,3,0,5,-1)$, cuenta cuántos elementos son positivos.

\begin{verbatim}
import numpy as np

v = np.array([-2, 3, 0, 5, -1])

contador = 0

for i in range(len(v)):
    if v[i] > 0:
        contador = contador + 1

print("Cantidad de positivos:", contador)
\end{verbatim}

% ==========================================================
\section*{Ejercicio 3 (for): Producto de todas las entradas}

Dado el vector $v=(1,2,3,-4,5)$, calcula el producto de todas sus entradas usando un bucle \texttt{for}.

\begin{verbatim}
import numpy as np

v = np.array([1, 2, 3, -4, 5])

producto = 1

for i in range(len(v)):
    producto = producto * v[i]

print(producto)
\end{verbatim}

% ==========================================================
\section*{Ejercicio 4 (while): Primer elemento negativo}

Para el vector $v=(1,2,3,-4,5)$, busca el primer elemento negativo usando un bucle \texttt{while}.

\begin{verbatim}
import numpy as np

v = np.array([1, 2, 3, -4, 5])

i = 0

while i < len(v):
    if v[i] < 0:
        print("Primer negativo:", v[i])
        break
    i = i + 1
\end{verbatim}

% ==========================================================
\section*{Ejercicio 5 (for): Suma de todos los elementos de una matriz}

Dada la matriz
\[
A=\begin{pmatrix}
1 & 2 \\
3 & 4
\end{pmatrix},
\]
calcula la suma de todos sus elementos usando bucles \texttt{for}.

\begin{verbatim}
import numpy as np

A = np.array([[1, 2],
              [3, 4]])

suma = 0

for i in range(2):
    for j in range(2):
        suma = suma + A[i, j]

print("Suma total:", suma)
\end{verbatim}

% ==========================================================
\section*{Ejercicio 6 (while): Primera columna con todos los elementos distintos de cero}

Dada la matriz
\[
B=\begin{pmatrix}
3 & \frac{2}{5} & 0 \\
0 & -1 & 4 \\
\pi & 0 & 1
\end{pmatrix},
\]
buscamos la primera columna cuyos elementos sean todos distintos de cero.

\begin{verbatim}
import numpy as np

B = np.array([
    [3, 2/5, 0],
    [0, -1, 4],
    [3.14, 0, 1]
])

j = 0

while j < 3:
    i = 0
    todos_distintos = True

    while i < 3:
        if B[i, j] == 0:
            todos_distintos = False
        i = i + 1

    if todos_distintos:
        print("Primera columna válida:", j)
        break

    j = j + 1
\end{verbatim}

% ==========================================================
\section*{Ejercicio 7 (for): Suma de las columnas}

Calculamos la suma de cada columna de la matriz anterior.

\begin{verbatim}
import numpy as np

B = np.array([
    [3, 2/5, 0],
    [0, -1, 4],
    [3.14, 0, 1]
])

for j in range(3):
    suma = 0
    for i in range(3):
        suma = suma + B[i, j]
    print("Suma columna", j, "=", suma)
\end{verbatim}

% ==========================================================
\section*{Ejercicio 8: Método de Jacobi (versión muy básica)}

Implementamos el método de Jacobi de la forma más simple posible.

\begin{verbatim}
import numpy as np

def jacobi(N, A, b, x, d):

    n = len(x)

    for k in range(N):

        x_nuevo = np.zeros(n)

        for i in range(n):
            suma = 0
            for j in range(n):
                if j != i:
                    suma = suma + A[i, j] * x[j]

            x_nuevo[i] = (b[i] - suma) / A[i, i]

        diferencia = 0
        for i in range(n):
            diferencia = diferencia + (x_nuevo[i] - x[i])**2

        if diferencia < d*d:
            break

        x = x_nuevo

    return x
\end{verbatim}

\end{document}

